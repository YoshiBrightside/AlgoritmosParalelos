% Algoritmos Paralelos 2019-2
% Plantilla para reportes de prácticas.

\documentclass{article}

\usepackage[utf8]{inputenc}
\usepackage{listings}

\title{Algoritmos Paralelos: Práctica 2}
\author{Pedrero Gómez Joshua Jair}
\date{15 de Febrero del 2019}

\begin{document}

\maketitle

\section{Descripción del programa}

La práctica 2 requirió de la creación de un único programa solamente:

\begin{enumerate}
	\item {\bf Practica02.c} \linebreak
	Programa que, recibiendo un número de elementos, los ordena utilizando un algoritmo de ordenamiento paralelo.
	Nótese que el n puede ser un numero arbitrario entre 0 y 100, sin necesidad de ser potencia de 2. Ésto, optando por un punto extra ofrecido dentro de la práctica.
\end{enumerate}

\section{Entrada y ejecución}

(Igualmente descrito dentro del archivo readme de la práctica 2).

\begin{enumerate}
	\item Ejecutar el archivo después de haber sido compilado, acompañado de el numero de elementos
	en el arreglo, y a continuacion dichos numeros correspondientes al arreglo.
	
	Ejemplo: 
	
	\begin{lstlisting}[language=c]
	./<programa> 5 1 2 3 4 5
	\end{lstlisting}
\end{enumerate}




\end{document}