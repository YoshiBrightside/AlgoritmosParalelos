% Algoritmos Paralelos 2019-2
% Plantilla para reportes de prácticas.

\documentclass{article}

\usepackage[utf8]{inputenc}
\usepackage{listings}

\title{Algoritmos Paralelos: Práctica 3}
\author{Pedrero Gómez Joshua Jair}
\date{1 de Marzo del 2019}

\begin{document}

\maketitle

\section{Descripción del programa}

La práctica 3 requirió de la creación de un único programa solamente:

\begin{enumerate}
	\item {\bf Practica03.c} \linebreak
	Programa que, recibiendo un número de procesadores dados por el usuario, aproxima el área bajo la curva de la siguiente función dada usando Sumas de Riemman:
	
	$$100-(x-10)^4+50(x-10)^2-8x$$
	
	$p$ puede ser un numero arbitrario entre 0 y 100, y cada procesador actuará de forma paralela haciendo 5 rectángulos cada uno. Entonces, habrá una razón de $5p$ rectángulos haciendo la aproximación de la función.
\end{enumerate}

\section{Entrada y ejecución}

(Igualmente descrito dentro del archivo readme de la práctica 3).

\begin{enumerate}
	\item Ejecutar el archivo después de haber sido compilado, acompañado del número de procesadores.
	
	Ejemplo: 
	
	\begin{lstlisting}[language=c]
	./<programa> 4
	\end{lstlisting}
\end{enumerate}




\end{document}