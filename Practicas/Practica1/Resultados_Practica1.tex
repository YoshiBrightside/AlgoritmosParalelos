% Algoritmos Paralelos 2019-2
% Plantilla para reportes de prácticas.

\documentclass{article}

\usepackage[utf8]{inputenc}
\usepackage{listings}

\title{Algoritmos Paralelos: Práctica 1}
\author{Pedrero Gómez Joshua Jair}
\date{3 de Febrero del 2019}

\begin{document}

\maketitle

\section{Descripción de los Programas}

Dentro de la Práctica 1 fue requerida la creación de cuatro programas:
\begin{enumerate}
	\item \textbf{Practica01.c} \linebreak 
	Programa que ejecuta una misma suma utilizando una diferente cantidad de hilos, los cuales simulan procesadores. Después, se entrega el tiempo transcurrido durante dicha suma.
	\item \textbf{Speedup.c} \linebreak
	Programa que toma el tiempo transcurrido en un algoritmo secuencial,
	en uno paralelo, y la cantidad de procesadores usados en dicho algoritmo paralelo. Devuelve cuantas veces es más rápido el algoritmo paralelo a comparación del secuencial.
	\item \textbf{Eficiencia.c} \linebreak
	Programa que recibe la aceleración (speedup) de una cantidad de procesadores, y mide que tan eficientemente se están utilizando esos procesadores a comparación de un algoritmo secuencial.
	\item \textbf{Fraccion\_serial.c} \linebreak
	Programa que, recibiendo la aceleración (speedup) y la cantidad de procesadores, determina la parte del código que es inherentemente secuencial.
	
\end{enumerate}

\section{Tabla de Resultados}

Ahora sí, aquí llegó la puerca.

\begin{tabular}
	
\end{tabular}

\begin{lstlisting}[language=Haskell]
Prelude> 3+2
5
\end{lstlisting}


\end{document}