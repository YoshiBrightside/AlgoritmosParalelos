% Algoritmos Paralelos 2019-2
% Plantilla para reportes de prácticas.

\documentclass{article}

\usepackage[utf8]{inputenc}
\usepackage{listings}

\title{Algoritmos Paralelos: Práctica 1}
\author{Pedrero Gómez Joshua Jair}
\date{4 de Febrero del 2019}

\begin{document}

\maketitle

\section{Descripción de los Programas}

Dentro de la Práctica 1 fue requerida la creación de cuatro programas:
\begin{enumerate}
	\item \textbf{Practica01.c} \linebreak 
	Programa que ejecuta una misma suma utilizando una diferente cantidad de hilos, los cuales simulan procesadores. Después, se entrega el tiempo transcurrido durante dicha suma. Para tener una mayor precisión en el tiempo de ejecución, se calcula el tiempo 10 veces con la misma cantidad de hilos, y después se hace un promedio con sus resultados.
	\item \textbf{Speedup.c} \linebreak
	Programa que toma el tiempo transcurrido en un algoritmo secuencial,
	en uno paralelo, y la cantidad de procesadores usados en dicho algoritmo paralelo. Devuelve cuantas veces es más rápido el algoritmo paralelo a comparación del secuencial.
	\item \textbf{Eficiencia.c} \linebreak
	Programa que recibe la aceleración (speedup) de una cantidad de procesadores, y mide que tan eficientemente se están utilizando esos procesadores a comparación de un algoritmo secuencial.
	\item \textbf{Fraccion\_serial.c} \linebreak
	Programa que, recibiendo la aceleración (speedup) y la cantidad de procesadores, determina la parte del código que es inherentemente secuencial.
	
\end{enumerate}

\pagebreak

\section{Tabla de Resultados}

\begin{center}
	\begin{tabular} {||r|c|c|c|c||}
		\hline \hline
		\bf{Procesos} & \bf{Tiempo de Ejecución} & \bf{Speedup} & \bf{Eficiencia} & \bf{Fracción Serial} \\ \hline
		1 & 4188587 ms & 1 & 1 & - \\ \hline
		2 & 2099287 ms & 1.995243 & 0.997621 & 0.501192 \\ \hline
		4 & 1877365 ms & 2.231099 & 0.557775 & 0.448210 \\ \hline
		6 & 1872573 ms & 2.236808 & 0.372801 & 0.447066 \\ \hline
		8 & 1868530 ms & 2.241648 & 0.280206 & 0.446100  \\ \hline
		10 & 1860565 ms & 2.251245 & 0.225125 & 0.444199 \\ \hline
		20 & 1867824 ms & 2.242496 & 0.112125 & 0.445932 \\ \hline
		50 & 1875525 ms & 2.233288 & 0.044666 & 0.447770 \\ \hline
		100 & 1876120 ms & 2.232579 & 0.022326 & 0.447912 \\ \hline
		\hline
	\end{tabular}
\end{center}


\end{document}